\documentclass{report}
\usepackage{filecontents}

\usepackage[utf8]{inputenc}
\usepackage[T1]{fontenc}
\usepackage[francais]{babel}
\usepackage{listings}
\usepackage[a4paper]{geometry}
\usepackage{graphicx}
\usepackage[export]{adjustbox}
\usepackage{titlesec}
\usepackage{color}
\usepackage[toc, page]{appendix}
\usepackage{url}

\definecolor{xcodekw}{rgb}{0.75, 0.22, 0.60}
\definecolor{xcodestr}{rgb}{0.89, 0.27, 0.30}
\definecolor{xcodecmt}{rgb}{0.31, 0.73, 0.35}

\titleformat{\chapter}[display]
  {\centering\normalfont\huge\bfseries}
  {\chaptertitlename\ \thechapter}
  {20pt}
  {\Huge}

\geometry{hscale=0.75,vscale=0.85,centering}

\renewcommand{\thesection}{\arabic{section}}
\renewcommand\appendixtocname{Annexes}
\renewcommand\appendixname{Annexes}
\renewcommand\appendixpagename{Annexes}

\title{
\parbox{15cm}
{ %\includegraphics[width=4cm]{foxhound.png} \\
  \vspace{3cm}
	\begin{center}\sf\bfseries\Huge
		\rule{15cm}{1pt}
		\medskip
		Beer Collection \\
		\huge Rapport de Sprint 1\\
		\vspace{.5cm}
		\rule{15cm}{1pt}
	\end{center}
	\vspace{3cm}
 }}
\author{Michaël \bsc{Riffon}, Maxime \bsc{Grenier}, Christophe \bsc{Henry}\\
Rémy \bsc{Voet}, Cyril \bsc{Pierret}, Samuel \bsc{Monroe}, Florian \bsc{Faingnaert}}

\date{7 Octobre 2015}

\begin{document}

\maketitle

\newpage
\thispagestyle{empty}
\mbox{}

\section{Introduction}

  Ce rapport présente l'avancement du projet \textbf{Beer Collection} au terme de ce premier sprint.\\
  Nous allons exposer les éléments qui ont été accomplis, notre état d'avancement par rapport aux prévisions et la vélocité estimée
  de notre équipe par rapport aux points des Stories completées.\\

\section{User et Technical Stories terminées}

  Notre définition de "fini" impliquera que la story soit fonctionnelle à 100%, et disposant d'un design suffisamment abouti pour ne nécéssiter que quelques modifications finales pour assurer l'homogénéité visuelle du projet.\\

  Nous n'avons donc aucune User Story réellement terminée selon nos exigences.

\section{Avancement par rapport aux prévisions}

  Malgré l'absence de User Stories terminées, nous avons atteint un état d'avancement pas si éloigné des objectifs initiaux.\\

  Beaucoup de temps a surtout été consacré à l'apprentissage des technologies que nous utiliserons pour la suite du projet, ainsi
  qu'à la mise en place des environnements de travail, de l'adaptation à la méthode SCRUM, ainsi qu'au rythme de travail à avoir.\\

  Nous estimons notre avancement à 70% sur la prévision initiale.\\

\section{Vélocité de l'équipe sur le sprint}

  Cette estimation est un peu biaisée car l'estimation des points à donner aux Stories s'est révélée assez peu précise.\\

  Cependant, si on considère l'attribution initiale des points, nous en sommes à 
